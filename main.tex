\documentclass[9pt]{extarticle}  % Usa extarticle para poder elegir 9pt.
\usepackage[utf8]{inputenc}
\usepackage[left=0.5in, right=0.5in, top=0.5in, bottom=0.8in]{geometry}
\usepackage{tabularx}
\usepackage{fancyhdr}
\pagestyle{fancy}
\fancyhf{}
\fancyhead[C]{Apunte para el primer parcial de AED hecho por Gaspi}
\usepackage{enumitem}
\usepackage{xcolor}
\usepackage{amssymb}
\usepackage{cancel}
\documentclass[9pt]{extarticle}  % Usa extarticle para poder elegir 9pt.
\usepackage[utf8]{inputenc}
\usepackage[left=0.5in, right=0.5in, top=0.5in, bottom=0.8in]{geometry}
\usepackage{tabularx}
\usepackage{fancyhdr}
\pagestyle{fancy}
\fancyhf{}
\fancyhead[C]{Apunte para el primer parcial de AED hecho por Gaspi}
\usepackage{enumitem}
\usepackage{xcolor}
\usepackage{amssymb}
\usepackage{cancel}
\documentclass[9pt]{extarticle}  % Usa extarticle para poder elegir 9pt.
\usepackage[utf8]{inputenc}
\usepackage[left=0.5in, right=0.5in, top=0.5in, bottom=0.8in]{geometry}
\usepackage{tabularx}
\usepackage{fancyhdr}
\pagestyle{fancy}
\fancyhf{}
\fancyhead[C]{Apunte para el primer parcial de AED hecho por Gaspi}
\usepackage{enumitem}
\usepackage{xcolor}
\usepackage{amssymb}
\usepackage{cancel}
\documentclass[9pt]{extarticle}  % Usa extarticle para poder elegir 9pt.
\usepackage[utf8]{inputenc}
\usepackage[left=0.5in, right=0.5in, top=0.5in, bottom=0.8in]{geometry}
\usepackage{tabularx}
\usepackage{fancyhdr}
\pagestyle{fancy}
\fancyhf{}
\fancyhead[C]{Apunte para el primer parcial de AED hecho por Gaspi}
\usepackage{enumitem}
\usepackage{xcolor}
\usepackage{amssymb}
\usepackage{cancel}
\input{AEDmacros}

% Reducir tamaño de fuente para tablas
\newcommand{\smalltable}{\fontsize{8pt}{10pt}\selectfont}
\renewcommand{\arraystretch}{0.8}  % Reduce el espacio vertical en las tablas.


\begin{document}

\section*{\tiny{Tipos complejos}}
    \vspace{-0.2cm}
\subsection*{\tiny{seq$<T>$: secuencia de tipo T}}
\vspace{-0.2cm}
\noindent\smalltable
\begin{tabularx}{\linewidth}{|X|X|}
\hline
crear & \( \langle \rangle \), \( \langle x, y, z \rangle \) \\
\hline
tamaño & \( |s| \), \text{length}(s) \\
\hline
pertenece & \( i \in s \) \\
\hline
ver posición & \( s[i] \) \\
\hline
cabeza & \text{head}(s) \\
\hline
cola & \text{tail}(s) \\
\hline
concatenar & \text{concat}(s1, s2), \( s1 + s2 \) \\
\hline
subsecuencia & \text{subseq}(s, i, j), \( s[i..j] \) \\
\hline
setear posición & \text{setAt}(s, i, \text{val}) \\
\hline
\end{tabularx}
\vspace{-0.5cm}
\subsection*{\tiny{conj$<T>$: conjunto de tipo T}}
\vspace{-0.2cm}
\noindent\smalltable
\vspace{-0.4cm}
\begin{tabularx}{\linewidth}{|X|X|}
\hline
crear & \( \{\} \), \( \{x, y, z\} \) \\
\hline
tamaño & \( |c| \), \text{length}(c) \\
\hline
pertenece & \( i \in c \) \\
\hline
unión & \( c1 \cup c2 \) \\
\hline
intersección & \( c1 \cap c2 \) \\
\hline
diferencia & \( c1 - c2 \) \\
\hline
\end{tabularx}
\vspace{-0.1cm}
\subsection*{\tiny{dict$<$K, V$>$: diccionario que asocia claves de tipo K con valores de tipo V}}
\vspace{-0.2cm}
\noindent\smalltable
\vspace{-0.4cm}
\begin{tabularx}{\linewidth}{|X|X|}
\hline
crear & \( \{\} \), \( \{\text{``juan''} : 20, \text{``diego''} : 10\} \) \\
\hline
tamaño & \( |d| \), \text{length}(d) \\
\hline
pertenece (hay clave) & \( k \in d \) \\
\hline
valor & d[k] \\
\hline
setear valor & \text{setKey}(d, k, v) \\
\hline
eliminar valor & \text{delKey}(d, k) \\
\hline
\end{tabularx}
\vspace{-0.1cm}
\subsection*{\tiny{tupla$<$T1, ..., Tn$>$: tupla de tipos T1, ..., Tn}}
\vspace{-0.2cm}
\noindent\smalltable
\begin{tabularx}{\linewidth}{|X|X|}
\hline
crear & \( \langle x, y, z \rangle \) \\
\hline
campo & \( s[i] \) \\
\hline
\end{tabularx}
\vspace{-0.5cm}
\subsection*{\tiny{struct$<$campo1: T1, ..., campon: Tn$>$: tupla con nombres para los campos.}}
\vspace{-0.2cm}
\noindent\smalltable
\begin{tabularx}{\linewidth}{|X|X|}
\hline
crear & \( \langle x : 20, y : 10 \rangle \) \\
\hline
campo & \( s_x, s_y \) \\
\hline
\end{tabularx}
\vspace{-0.2cm}
\subsection*{\tiny{Precondición más débil (wp)}}
\vspace{-0.3cm}
\noindent\smalltable
\vspace{-0.2cm}
La \textit{precondición más débil} es la condición mínima necesaria antes de ejecutar una sentencia para garantizar que una postcondición dada se cumpla.
\subsection*{\tiny{Lógica de Hoare}}
\vspace{-0.3cm}
\noindent\smalltable
\vspace{-0.1cm}
Una forma de razonar sobre la corrección de programas es a través de las triples de Hoare \(\{P\} C \{Q\}\). El objetivo es obtener una fórmula lógica \( \alpha \) tal que \( \alpha \) es verdadera si y solo si \(\{P\} C \{Q\}\) es verdadero.
\vspace{-0.5cm}
\paragraph*{Predicado def(E):}
Definición. Dada una expresión \( E \), llamamos \( \text{def}(E) \) a las condiciones necesarias para que \( E \) esté definida.
\vspace{-0.45cm}
\paragraph*{Predicado \( Q_{E}^x \):}
Definición. Dado un predicado \( Q \), el predicado \( Q_{E}^x \) se obtiene reemplazando...
\vspace{-0.4cm}
\paragraph*{Axiomas WP}
\begin{itemize}\vspace{-0.35cm}
    \item \textbf{Axioma 1.} \( \text{wp}(x := E, Q) \equiv \text{def}(E) \land Q_{E}^x \)\vspace{-0.35cm}
    \item \textbf{Axioma 2.} \( \text{wp}(\text{skip}, Q) \equiv Q \).
    \vspace{-0.33cm}
    \item \textbf{Axioma 3.} \( \text{wp}(S1; S2, Q) \equiv \text{wp}(S1,\text{wp}(S2, Q)) \).\vspace{-0.33cm}
    \item \textbf{Axioma 4.} Si \(S = \text{if } B \text{ then } S1 \text{ else } S2 \text{ endif}\), entonces
    \vspace{-0.31cm}
    \[
    \vspace{-0.37cm}
    \text{wp}(S, Q) \equiv \text{def}(B) \yLuego \left( (B \land \text{wp}(S1, Q)) \vee (\neg B \land \text{wp}(S2, Q)) \right)
    \]
     \item \textbf{Axioma 5.} \( \text{wp}(\text{while } B \text{ do } S \text{ endwhile}, Q) \equiv (\exists_{i \geq 0})({ H_i}(Q)) \)  no podemos usar mecanicamente el Axioma 5
para demostrar la corrección de un ciclo con una cantidad no
acotada a priori de iteraciones
\vspace{-0.34cm}
\end{itemize}
\subsection*{\tiny{Asignación a elementos de una secuencia}}
\noindent\smalltable
\vspace{-0.3cm}
Aplicando el Axioma 1, obtenemos:
\[ \text{wp}(b[i] := E, Q) \equiv \text{wp}(b := \text{setAt}(b, i, E), Q) \]
\[ \equiv (\text{def}(b) \land \text{def}(i)) \land 0 \leq i < |b| \land \text{def}(E) \land Q_{\text{setAt}(b,i,E)}^b \]

Dado que \( 0 \leq i, j < |b| \), se sabe que:
\[ \text{setAt}(b, i, E)[j] = 
\begin{cases} 
\vspace{-0.2cm}
E & \text{si } i = j \\
b[j] & \text{si } i \neq j 
\end{cases} 
\]

\vspace{-0.4cm}
\vspace{-0.5cm}
\paragraph*{Propiedades:}
\begin{itemize}
\vspace{-0.32cm}
    \item Monotonía: Si \( Q \) implica \( R \), entonces \( \text{wp}(S, Q) \) implica \( \text{wp}(S, R) \).
    \vspace{-0.32cm}
    \item Distributividad: \( \text{wp}(S, Q \land R) \) equivale a \( \text{wp}(S, Q) \land \text{wp}(S, R) \).
    \vspace{-0.32cm}
    \item Excluded Miracle: \( \text{wp}(S, \text{false}) \) equivale a \( \text{false} \).
\end{itemize}
\vspace{-0.67cm}
\subsection*{\tiny{Invariante de un ciclo}}
\noindent\smalltable
\vspace{-0.2cm}
\textbf{Definición.} Un predicado \( I \) es un invariante de un ciclo si:
\begin{enumerate}
    \item \( I \) vale antes de comenzar el ciclo, y
    \vspace{-0.32cm}
    \item Si vale \( I \land B \) al comenzar una iteración arbitraria, entonces sigue valiendo \( I \) al finalizar la ejecución del cuerpo del ciclo.
\end{enumerate}
\vspace{-0.22cm}
Un invariante describe un estado que se satisface cada vez que comienza la ejecución del cuerpo de un ciclo y también se cumple cuando la ejecución del cuerpo del ciclo concluye.
\vspace{-0.2cm}
\textbf{Teorema del Invariante.} Si existe un predicado \( I \) tal que:
\begin{enumerate}
    \item \( P_C \Rightarrow I \),
    \item $ \{ I \wedge B \} S \{ I \} \iff (I \wedge B) \implies wp(S, I) $
    \item \(I \land \neg B \Rightarrow Q_C\),
\end{enumerate}
entonces \( \{P_C\} \text{ while } B \text{ do } S \text{ endwhile } \{Q_C\} \) es válida.
\subsection*{\tiny{Función variante}}
\noindent\smalltable
La función variante representa una cantidad que se va
reduciendo
\begin{itemize}
    \item {Si existe una función variante \(f_v\) tal que:
\begin{enumerate}\item $ \{ I \wedge B \wedge f_v = v_0 \} S \{ f_v < v_0 \} \iff (I \wedge B \wedge f_v = v_0) \implies wp(S, f_v < v_0) $

    \item Si \(I \land f_v \leq 0\) implica \(\neg B\)
\end{enumerate}}
\end{itemize}
\subsection*{\tiny{Correctitud de un Programa Completo}}
\noindent\smalltable
Para demostrar la correctitud de un programa completo utilizando la lógica de Hoare, seguimos estos pasos:

\begin{enumerate}
    \item \textbf{Código antes del ciclo:} Debemos demostrar que las precondiciones implican la precondición más débil del código previo al ciclo, es decir,
    \[ Pre \Rightarrow \text{wp}(Código\_previo, P_c) \]
    
    \item \textbf{Correctitud del ciclo:} Utilizamos el teorema del invariante, ya que no podemos calcular la precondición más débil del ciclo en general. El teorema del invariante se expresa de la siguiente manera:
    
    \begin{enumerate}
        \item \[ P_c \Rightarrow I \]
        \item \[ (I \land B) \Rightarrow \text{wp}(Ciclo, I) \]
        \item \[ (I \land \neg B) \Rightarrow Q_c \]
    \end{enumerate}

    \item \textbf{Código posterior al ciclo:} Comprobamos que las postcondiciones del ciclo implican la precondición más débil del código posterior al ciclo, es decir,
    \[ Q_c \Rightarrow \text{wp}(Código\_posterior, Post) \]

    \item \textbf{Conclusión:} Si probamos estas tres cosas, podemos concluir, por corolario de monotonía, que el programa completo es correcto con respecto a la especificación:
    \[ Pre \Rightarrow \text{wp}(Programa\_completo, Post) \]
\end{enumerate}
\vspace{-0.6cm}

\subsection*{\tiny{Especificación y Relaciones de Fuerza}}

\noindent\textbf{Especificación:} \\
La especificación define qué es lo que debe hacer un algoritmo, en términos de relación entre sus entradas y sus salidas, sin determinar cómo lo hace. Las relaciones de fuerza entre especificaciones son importantes para entender cuán restrictiva o general es una especificación con respecto a otra.

\noindent\textbf{Subespecificación:} \\
Implica otorgar una precondición más restrictiva o una postcondición más débil que lo deducido del enunciado del problema. Una precondición más restrictiva excluye casos de entrada posibles. Una postcondición más débil permite soluciones no deseadas.

\noindent\textbf{Sobreespecificación:} \\
Consiste en proporcionar una postcondición más restrictiva o una precondición más débil que lo necesario. Una precondición más débil obliga al algoritmo a considerar casos innecesarios. Una postcondición más restrictiva limita las posibles soluciones.

\noindent\textbf{Relaciones de Fuerza:} \\
Decimos que \(A\) es más fuerte que \(B\) cuando \(A \rightarrow B\) es una tautología. También podemos afirmar que \(A\) fuerza a \(B\) o que \(B\) es más débil que \(A\).

\noindent\textbf{Ejemplos:}
\begin{itemize}
\vspace{-0.3cm}
    \item \(p \land q\) es más fuerte que \(p\).
    \vspace{-0.3cm}
    \item \(p \lor q\) no es más fuerte que \(p\).
\vspace{-0.3cm}
    \item \(p\) es más fuerte que \(p \rightarrow q\).
    \vspace{-0.3cm}
    \item \(p\) no es más fuerte que \(q\).
    \vspace{-0.3cm}
    \item \textbf{False} es la fórmula más fuerte de todas.
    \vspace{-0.3cm}
    \item \textbf{True} es la fórmula más débil de todas.
    \vspace{-0.3cm}
\end{itemize}
\subsection*{\tiny {Truco algebraico para sacar términos del último rango}}
    \vspace{-0.2cm}
(\forall k:\ent) \, (i \leq k < |s| \implicaLuego p(k)) \equiv \land p(i) (\forall k:\ent ) (i < k < |s|) \implicaLuego (p(k))

Por otro lado, para sumatorias, tenemos que:
\sum_{k=i}^{|s|-1} p(k) = p(i) + \sum_{k=i+1}^{|s|-1} p(k)
\\ ejemplo:
\[ \sum_{j=0}^{5} j = \sum_{j=0}^{4} j + 5 \] \\
Esto nos permite sacar el término correspondiente a \( i \) fuera de la sumatoria.




\end{document}




% Reducir tamaño de fuente para tablas
\newcommand{\smalltable}{\fontsize{8pt}{10pt}\selectfont}
\renewcommand{\arraystretch}{0.8}  % Reduce el espacio vertical en las tablas.


\begin{document}

\section*{\tiny{Tipos complejos}}
    \vspace{-0.2cm}
\subsection*{\tiny{seq$<T>$: secuencia de tipo T}}
\vspace{-0.2cm}
\noindent\smalltable
\begin{tabularx}{\linewidth}{|X|X|}
\hline
crear & \( \langle \rangle \), \( \langle x, y, z \rangle \) \\
\hline
tamaño & \( |s| \), \text{length}(s) \\
\hline
pertenece & \( i \in s \) \\
\hline
ver posición & \( s[i] \) \\
\hline
cabeza & \text{head}(s) \\
\hline
cola & \text{tail}(s) \\
\hline
concatenar & \text{concat}(s1, s2), \( s1 + s2 \) \\
\hline
subsecuencia & \text{subseq}(s, i, j), \( s[i..j] \) \\
\hline
setear posición & \text{setAt}(s, i, \text{val}) \\
\hline
\end{tabularx}
\vspace{-0.5cm}
\subsection*{\tiny{conj$<T>$: conjunto de tipo T}}
\vspace{-0.2cm}
\noindent\smalltable
\vspace{-0.4cm}
\begin{tabularx}{\linewidth}{|X|X|}
\hline
crear & \( \{\} \), \( \{x, y, z\} \) \\
\hline
tamaño & \( |c| \), \text{length}(c) \\
\hline
pertenece & \( i \in c \) \\
\hline
unión & \( c1 \cup c2 \) \\
\hline
intersección & \( c1 \cap c2 \) \\
\hline
diferencia & \( c1 - c2 \) \\
\hline
\end{tabularx}
\vspace{-0.1cm}
\subsection*{\tiny{dict$<$K, V$>$: diccionario que asocia claves de tipo K con valores de tipo V}}
\vspace{-0.2cm}
\noindent\smalltable
\vspace{-0.4cm}
\begin{tabularx}{\linewidth}{|X|X|}
\hline
crear & \( \{\} \), \( \{\text{``juan''} : 20, \text{``diego''} : 10\} \) \\
\hline
tamaño & \( |d| \), \text{length}(d) \\
\hline
pertenece (hay clave) & \( k \in d \) \\
\hline
valor & d[k] \\
\hline
setear valor & \text{setKey}(d, k, v) \\
\hline
eliminar valor & \text{delKey}(d, k) \\
\hline
\end{tabularx}
\vspace{-0.1cm}
\subsection*{\tiny{tupla$<$T1, ..., Tn$>$: tupla de tipos T1, ..., Tn}}
\vspace{-0.2cm}
\noindent\smalltable
\begin{tabularx}{\linewidth}{|X|X|}
\hline
crear & \( \langle x, y, z \rangle \) \\
\hline
campo & \( s[i] \) \\
\hline
\end{tabularx}
\vspace{-0.5cm}
\subsection*{\tiny{struct$<$campo1: T1, ..., campon: Tn$>$: tupla con nombres para los campos.}}
\vspace{-0.2cm}
\noindent\smalltable
\begin{tabularx}{\linewidth}{|X|X|}
\hline
crear & \( \langle x : 20, y : 10 \rangle \) \\
\hline
campo & \( s_x, s_y \) \\
\hline
\end{tabularx}
\vspace{-0.2cm}
\subsection*{\tiny{Precondición más débil (wp)}}
\vspace{-0.3cm}
\noindent\smalltable
\vspace{-0.2cm}
La \textit{precondición más débil} es la condición mínima necesaria antes de ejecutar una sentencia para garantizar que una postcondición dada se cumpla.
\subsection*{\tiny{Lógica de Hoare}}
\vspace{-0.3cm}
\noindent\smalltable
\vspace{-0.1cm}
Una forma de razonar sobre la corrección de programas es a través de las triples de Hoare \(\{P\} C \{Q\}\). El objetivo es obtener una fórmula lógica \( \alpha \) tal que \( \alpha \) es verdadera si y solo si \(\{P\} C \{Q\}\) es verdadero.
\vspace{-0.5cm}
\paragraph*{Predicado def(E):}
Definición. Dada una expresión \( E \), llamamos \( \text{def}(E) \) a las condiciones necesarias para que \( E \) esté definida.
\vspace{-0.45cm}
\paragraph*{Predicado \( Q_{E}^x \):}
Definición. Dado un predicado \( Q \), el predicado \( Q_{E}^x \) se obtiene reemplazando...
\vspace{-0.4cm}
\paragraph*{Axiomas WP}
\begin{itemize}\vspace{-0.35cm}
    \item \textbf{Axioma 1.} \( \text{wp}(x := E, Q) \equiv \text{def}(E) \land Q_{E}^x \)\vspace{-0.35cm}
    \item \textbf{Axioma 2.} \( \text{wp}(\text{skip}, Q) \equiv Q \).
    \vspace{-0.33cm}
    \item \textbf{Axioma 3.} \( \text{wp}(S1; S2, Q) \equiv \text{wp}(S1,\text{wp}(S2, Q)) \).\vspace{-0.33cm}
    \item \textbf{Axioma 4.} Si \(S = \text{if } B \text{ then } S1 \text{ else } S2 \text{ endif}\), entonces
    \vspace{-0.31cm}
    \[
    \vspace{-0.37cm}
    \text{wp}(S, Q) \equiv \text{def}(B) \yLuego \left( (B \land \text{wp}(S1, Q)) \vee (\neg B \land \text{wp}(S2, Q)) \right)
    \]
     \item \textbf{Axioma 5.} \( \text{wp}(\text{while } B \text{ do } S \text{ endwhile}, Q) \equiv (\exists_{i \geq 0})({ H_i}(Q)) \)  no podemos usar mecanicamente el Axioma 5
para demostrar la corrección de un ciclo con una cantidad no
acotada a priori de iteraciones
\vspace{-0.34cm}
\end{itemize}
\subsection*{\tiny{Asignación a elementos de una secuencia}}
\noindent\smalltable
\vspace{-0.3cm}
Aplicando el Axioma 1, obtenemos:
\[ \text{wp}(b[i] := E, Q) \equiv \text{wp}(b := \text{setAt}(b, i, E), Q) \]
\[ \equiv (\text{def}(b) \land \text{def}(i)) \land 0 \leq i < |b| \land \text{def}(E) \land Q_{\text{setAt}(b,i,E)}^b \]

Dado que \( 0 \leq i, j < |b| \), se sabe que:
\[ \text{setAt}(b, i, E)[j] = 
\begin{cases} 
\vspace{-0.2cm}
E & \text{si } i = j \\
b[j] & \text{si } i \neq j 
\end{cases} 
\]

\vspace{-0.4cm}
\vspace{-0.5cm}
\paragraph*{Propiedades:}
\begin{itemize}
\vspace{-0.32cm}
    \item Monotonía: Si \( Q \) implica \( R \), entonces \( \text{wp}(S, Q) \) implica \( \text{wp}(S, R) \).
    \vspace{-0.32cm}
    \item Distributividad: \( \text{wp}(S, Q \land R) \) equivale a \( \text{wp}(S, Q) \land \text{wp}(S, R) \).
    \vspace{-0.32cm}
    \item Excluded Miracle: \( \text{wp}(S, \text{false}) \) equivale a \( \text{false} \).
\end{itemize}
\vspace{-0.67cm}
\subsection*{\tiny{Invariante de un ciclo}}
\noindent\smalltable
\vspace{-0.2cm}
\textbf{Definición.} Un predicado \( I \) es un invariante de un ciclo si:
\begin{enumerate}
    \item \( I \) vale antes de comenzar el ciclo, y
    \vspace{-0.32cm}
    \item Si vale \( I \land B \) al comenzar una iteración arbitraria, entonces sigue valiendo \( I \) al finalizar la ejecución del cuerpo del ciclo.
\end{enumerate}
\vspace{-0.22cm}
Un invariante describe un estado que se satisface cada vez que comienza la ejecución del cuerpo de un ciclo y también se cumple cuando la ejecución del cuerpo del ciclo concluye.
\vspace{-0.2cm}
\textbf{Teorema del Invariante.} Si existe un predicado \( I \) tal que:
\begin{enumerate}
    \item \( P_C \Rightarrow I \),
    \item $ \{ I \wedge B \} S \{ I \} \iff (I \wedge B) \implies wp(S, I) $
    \item \(I \land \neg B \Rightarrow Q_C\),
\end{enumerate}
entonces \( \{P_C\} \text{ while } B \text{ do } S \text{ endwhile } \{Q_C\} \) es válida.
\subsection*{\tiny{Función variante}}
\noindent\smalltable
La función variante representa una cantidad que se va
reduciendo
\begin{itemize}
    \item {Si existe una función variante \(f_v\) tal que:
\begin{enumerate}\item $ \{ I \wedge B \wedge f_v = v_0 \} S \{ f_v < v_0 \} \iff (I \wedge B \wedge f_v = v_0) \implies wp(S, f_v < v_0) $

    \item Si \(I \land f_v \leq 0\) implica \(\neg B\)
\end{enumerate}}
\end{itemize}
\subsection*{\tiny{Correctitud de un Programa Completo}}
\noindent\smalltable
Para demostrar la correctitud de un programa completo utilizando la lógica de Hoare, seguimos estos pasos:

\begin{enumerate}
    \item \textbf{Código antes del ciclo:} Debemos demostrar que las precondiciones implican la precondición más débil del código previo al ciclo, es decir,
    \[ Pre \Rightarrow \text{wp}(Código\_previo, P_c) \]
    
    \item \textbf{Correctitud del ciclo:} Utilizamos el teorema del invariante, ya que no podemos calcular la precondición más débil del ciclo en general. El teorema del invariante se expresa de la siguiente manera:
    
    \begin{enumerate}
        \item \[ P_c \Rightarrow I \]
        \item \[ (I \land B) \Rightarrow \text{wp}(Ciclo, I) \]
        \item \[ (I \land \neg B) \Rightarrow Q_c \]
    \end{enumerate}

    \item \textbf{Código posterior al ciclo:} Comprobamos que las postcondiciones del ciclo implican la precondición más débil del código posterior al ciclo, es decir,
    \[ Q_c \Rightarrow \text{wp}(Código\_posterior, Post) \]

    \item \textbf{Conclusión:} Si probamos estas tres cosas, podemos concluir, por corolario de monotonía, que el programa completo es correcto con respecto a la especificación:
    \[ Pre \Rightarrow \text{wp}(Programa\_completo, Post) \]
\end{enumerate}
\vspace{-0.6cm}

\subsection*{\tiny{Especificación y Relaciones de Fuerza}}

\noindent\textbf{Especificación:} \\
La especificación define qué es lo que debe hacer un algoritmo, en términos de relación entre sus entradas y sus salidas, sin determinar cómo lo hace. Las relaciones de fuerza entre especificaciones son importantes para entender cuán restrictiva o general es una especificación con respecto a otra.

\noindent\textbf{Subespecificación:} \\
Implica otorgar una precondición más restrictiva o una postcondición más débil que lo deducido del enunciado del problema. Una precondición más restrictiva excluye casos de entrada posibles. Una postcondición más débil permite soluciones no deseadas.

\noindent\textbf{Sobreespecificación:} \\
Consiste en proporcionar una postcondición más restrictiva o una precondición más débil que lo necesario. Una precondición más débil obliga al algoritmo a considerar casos innecesarios. Una postcondición más restrictiva limita las posibles soluciones.

\noindent\textbf{Relaciones de Fuerza:} \\
Decimos que \(A\) es más fuerte que \(B\) cuando \(A \rightarrow B\) es una tautología. También podemos afirmar que \(A\) fuerza a \(B\) o que \(B\) es más débil que \(A\).

\noindent\textbf{Ejemplos:}
\begin{itemize}
\vspace{-0.3cm}
    \item \(p \land q\) es más fuerte que \(p\).
    \vspace{-0.3cm}
    \item \(p \lor q\) no es más fuerte que \(p\).
\vspace{-0.3cm}
    \item \(p\) es más fuerte que \(p \rightarrow q\).
    \vspace{-0.3cm}
    \item \(p\) no es más fuerte que \(q\).
    \vspace{-0.3cm}
    \item \textbf{False} es la fórmula más fuerte de todas.
    \vspace{-0.3cm}
    \item \textbf{True} es la fórmula más débil de todas.
    \vspace{-0.3cm}
\end{itemize}
\subsection*{\tiny {Truco algebraico para sacar términos del último rango}}
    \vspace{-0.2cm}
(\forall k:\ent) \, (i \leq k < |s| \implicaLuego p(k)) \equiv \land p(i) (\forall k:\ent ) (i < k < |s|) \implicaLuego (p(k))

Por otro lado, para sumatorias, tenemos que:
\sum_{k=i}^{|s|-1} p(k) = p(i) + \sum_{k=i+1}^{|s|-1} p(k)
\\ ejemplo:
\[ \sum_{j=0}^{5} j = \sum_{j=0}^{4} j + 5 \] \\
Esto nos permite sacar el término correspondiente a \( i \) fuera de la sumatoria.




\end{document}




% Reducir tamaño de fuente para tablas
\newcommand{\smalltable}{\fontsize{8pt}{10pt}\selectfont}
\renewcommand{\arraystretch}{0.8}  % Reduce el espacio vertical en las tablas.


\begin{document}

\section*{\tiny{Tipos complejos}}
    \vspace{-0.2cm}
\subsection*{\tiny{seq$<T>$: secuencia de tipo T}}
\vspace{-0.2cm}
\noindent\smalltable
\begin{tabularx}{\linewidth}{|X|X|}
\hline
crear & \( \langle \rangle \), \( \langle x, y, z \rangle \) \\
\hline
tamaño & \( |s| \), \text{length}(s) \\
\hline
pertenece & \( i \in s \) \\
\hline
ver posición & \( s[i] \) \\
\hline
cabeza & \text{head}(s) \\
\hline
cola & \text{tail}(s) \\
\hline
concatenar & \text{concat}(s1, s2), \( s1 + s2 \) \\
\hline
subsecuencia & \text{subseq}(s, i, j), \( s[i..j] \) \\
\hline
setear posición & \text{setAt}(s, i, \text{val}) \\
\hline
\end{tabularx}
\vspace{-0.5cm}
\subsection*{\tiny{conj$<T>$: conjunto de tipo T}}
\vspace{-0.2cm}
\noindent\smalltable
\vspace{-0.4cm}
\begin{tabularx}{\linewidth}{|X|X|}
\hline
crear & \( \{\} \), \( \{x, y, z\} \) \\
\hline
tamaño & \( |c| \), \text{length}(c) \\
\hline
pertenece & \( i \in c \) \\
\hline
unión & \( c1 \cup c2 \) \\
\hline
intersección & \( c1 \cap c2 \) \\
\hline
diferencia & \( c1 - c2 \) \\
\hline
\end{tabularx}
\vspace{-0.1cm}
\subsection*{\tiny{dict$<$K, V$>$: diccionario que asocia claves de tipo K con valores de tipo V}}
\vspace{-0.2cm}
\noindent\smalltable
\vspace{-0.4cm}
\begin{tabularx}{\linewidth}{|X|X|}
\hline
crear & \( \{\} \), \( \{\text{``juan''} : 20, \text{``diego''} : 10\} \) \\
\hline
tamaño & \( |d| \), \text{length}(d) \\
\hline
pertenece (hay clave) & \( k \in d \) \\
\hline
valor & d[k] \\
\hline
setear valor & \text{setKey}(d, k, v) \\
\hline
eliminar valor & \text{delKey}(d, k) \\
\hline
\end{tabularx}
\vspace{-0.1cm}
\subsection*{\tiny{tupla$<$T1, ..., Tn$>$: tupla de tipos T1, ..., Tn}}
\vspace{-0.2cm}
\noindent\smalltable
\begin{tabularx}{\linewidth}{|X|X|}
\hline
crear & \( \langle x, y, z \rangle \) \\
\hline
campo & \( s[i] \) \\
\hline
\end{tabularx}
\vspace{-0.5cm}
\subsection*{\tiny{struct$<$campo1: T1, ..., campon: Tn$>$: tupla con nombres para los campos.}}
\vspace{-0.2cm}
\noindent\smalltable
\begin{tabularx}{\linewidth}{|X|X|}
\hline
crear & \( \langle x : 20, y : 10 \rangle \) \\
\hline
campo & \( s_x, s_y \) \\
\hline
\end{tabularx}
\vspace{-0.2cm}
\subsection*{\tiny{Precondición más débil (wp)}}
\vspace{-0.3cm}
\noindent\smalltable
\vspace{-0.2cm}
La \textit{precondición más débil} es la condición mínima necesaria antes de ejecutar una sentencia para garantizar que una postcondición dada se cumpla.
\subsection*{\tiny{Lógica de Hoare}}
\vspace{-0.3cm}
\noindent\smalltable
\vspace{-0.1cm}
Una forma de razonar sobre la corrección de programas es a través de las triples de Hoare \(\{P\} C \{Q\}\). El objetivo es obtener una fórmula lógica \( \alpha \) tal que \( \alpha \) es verdadera si y solo si \(\{P\} C \{Q\}\) es verdadero.
\vspace{-0.5cm}
\paragraph*{Predicado def(E):}
Definición. Dada una expresión \( E \), llamamos \( \text{def}(E) \) a las condiciones necesarias para que \( E \) esté definida.
\vspace{-0.45cm}
\paragraph*{Predicado \( Q_{E}^x \):}
Definición. Dado un predicado \( Q \), el predicado \( Q_{E}^x \) se obtiene reemplazando...
\vspace{-0.4cm}
\paragraph*{Axiomas WP}
\begin{itemize}\vspace{-0.35cm}
    \item \textbf{Axioma 1.} \( \text{wp}(x := E, Q) \equiv \text{def}(E) \land Q_{E}^x \)\vspace{-0.35cm}
    \item \textbf{Axioma 2.} \( \text{wp}(\text{skip}, Q) \equiv Q \).
    \vspace{-0.33cm}
    \item \textbf{Axioma 3.} \( \text{wp}(S1; S2, Q) \equiv \text{wp}(S1,\text{wp}(S2, Q)) \).\vspace{-0.33cm}
    \item \textbf{Axioma 4.} Si \(S = \text{if } B \text{ then } S1 \text{ else } S2 \text{ endif}\), entonces
    \vspace{-0.31cm}
    \[
    \vspace{-0.37cm}
    \text{wp}(S, Q) \equiv \text{def}(B) \yLuego \left( (B \land \text{wp}(S1, Q)) \vee (\neg B \land \text{wp}(S2, Q)) \right)
    \]
     \item \textbf{Axioma 5.} \( \text{wp}(\text{while } B \text{ do } S \text{ endwhile}, Q) \equiv (\exists_{i \geq 0})({ H_i}(Q)) \)  no podemos usar mecanicamente el Axioma 5
para demostrar la corrección de un ciclo con una cantidad no
acotada a priori de iteraciones
\vspace{-0.34cm}
\end{itemize}
\subsection*{\tiny{Asignación a elementos de una secuencia}}
\noindent\smalltable
\vspace{-0.3cm}
Aplicando el Axioma 1, obtenemos:
\[ \text{wp}(b[i] := E, Q) \equiv \text{wp}(b := \text{setAt}(b, i, E), Q) \]
\[ \equiv (\text{def}(b) \land \text{def}(i)) \land 0 \leq i < |b| \land \text{def}(E) \land Q_{\text{setAt}(b,i,E)}^b \]

Dado que \( 0 \leq i, j < |b| \), se sabe que:
\[ \text{setAt}(b, i, E)[j] = 
\begin{cases} 
\vspace{-0.2cm}
E & \text{si } i = j \\
b[j] & \text{si } i \neq j 
\end{cases} 
\]

\vspace{-0.4cm}
\vspace{-0.5cm}
\paragraph*{Propiedades:}
\begin{itemize}
\vspace{-0.32cm}
    \item Monotonía: Si \( Q \) implica \( R \), entonces \( \text{wp}(S, Q) \) implica \( \text{wp}(S, R) \).
    \vspace{-0.32cm}
    \item Distributividad: \( \text{wp}(S, Q \land R) \) equivale a \( \text{wp}(S, Q) \land \text{wp}(S, R) \).
    \vspace{-0.32cm}
    \item Excluded Miracle: \( \text{wp}(S, \text{false}) \) equivale a \( \text{false} \).
\end{itemize}
\vspace{-0.67cm}
\subsection*{\tiny{Invariante de un ciclo}}
\noindent\smalltable
\vspace{-0.2cm}
\textbf{Definición.} Un predicado \( I \) es un invariante de un ciclo si:
\begin{enumerate}
    \item \( I \) vale antes de comenzar el ciclo, y
    \vspace{-0.32cm}
    \item Si vale \( I \land B \) al comenzar una iteración arbitraria, entonces sigue valiendo \( I \) al finalizar la ejecución del cuerpo del ciclo.
\end{enumerate}
\vspace{-0.22cm}
Un invariante describe un estado que se satisface cada vez que comienza la ejecución del cuerpo de un ciclo y también se cumple cuando la ejecución del cuerpo del ciclo concluye.
\vspace{-0.2cm}
\textbf{Teorema del Invariante.} Si existe un predicado \( I \) tal que:
\begin{enumerate}
    \item \( P_C \Rightarrow I \),
    \item $ \{ I \wedge B \} S \{ I \} \iff (I \wedge B) \implies wp(S, I) $
    \item \(I \land \neg B \Rightarrow Q_C\),
\end{enumerate}
entonces \( \{P_C\} \text{ while } B \text{ do } S \text{ endwhile } \{Q_C\} \) es válida.
\subsection*{\tiny{Función variante}}
\noindent\smalltable
La función variante representa una cantidad que se va
reduciendo
\begin{itemize}
    \item {Si existe una función variante \(f_v\) tal que:
\begin{enumerate}\item $ \{ I \wedge B \wedge f_v = v_0 \} S \{ f_v < v_0 \} \iff (I \wedge B \wedge f_v = v_0) \implies wp(S, f_v < v_0) $

    \item Si \(I \land f_v \leq 0\) implica \(\neg B\)
\end{enumerate}}
\end{itemize}
\subsection*{\tiny{Correctitud de un Programa Completo}}
\noindent\smalltable
Para demostrar la correctitud de un programa completo utilizando la lógica de Hoare, seguimos estos pasos:

\begin{enumerate}
    \item \textbf{Código antes del ciclo:} Debemos demostrar que las precondiciones implican la precondición más débil del código previo al ciclo, es decir,
    \[ Pre \Rightarrow \text{wp}(Código\_previo, P_c) \]
    
    \item \textbf{Correctitud del ciclo:} Utilizamos el teorema del invariante, ya que no podemos calcular la precondición más débil del ciclo en general. El teorema del invariante se expresa de la siguiente manera:
    
    \begin{enumerate}
        \item \[ P_c \Rightarrow I \]
        \item \[ (I \land B) \Rightarrow \text{wp}(Ciclo, I) \]
        \item \[ (I \land \neg B) \Rightarrow Q_c \]
    \end{enumerate}

    \item \textbf{Código posterior al ciclo:} Comprobamos que las postcondiciones del ciclo implican la precondición más débil del código posterior al ciclo, es decir,
    \[ Q_c \Rightarrow \text{wp}(Código\_posterior, Post) \]

    \item \textbf{Conclusión:} Si probamos estas tres cosas, podemos concluir, por corolario de monotonía, que el programa completo es correcto con respecto a la especificación:
    \[ Pre \Rightarrow \text{wp}(Programa\_completo, Post) \]
\end{enumerate}
\vspace{-0.6cm}

\subsection*{\tiny{Especificación y Relaciones de Fuerza}}

\noindent\textbf{Especificación:} \\
La especificación define qué es lo que debe hacer un algoritmo, en términos de relación entre sus entradas y sus salidas, sin determinar cómo lo hace. Las relaciones de fuerza entre especificaciones son importantes para entender cuán restrictiva o general es una especificación con respecto a otra.

\noindent\textbf{Subespecificación:} \\
Implica otorgar una precondición más restrictiva o una postcondición más débil que lo deducido del enunciado del problema. Una precondición más restrictiva excluye casos de entrada posibles. Una postcondición más débil permite soluciones no deseadas.

\noindent\textbf{Sobreespecificación:} \\
Consiste en proporcionar una postcondición más restrictiva o una precondición más débil que lo necesario. Una precondición más débil obliga al algoritmo a considerar casos innecesarios. Una postcondición más restrictiva limita las posibles soluciones.

\noindent\textbf{Relaciones de Fuerza:} \\
Decimos que \(A\) es más fuerte que \(B\) cuando \(A \rightarrow B\) es una tautología. También podemos afirmar que \(A\) fuerza a \(B\) o que \(B\) es más débil que \(A\).

\noindent\textbf{Ejemplos:}
\begin{itemize}
\vspace{-0.3cm}
    \item \(p \land q\) es más fuerte que \(p\).
    \vspace{-0.3cm}
    \item \(p \lor q\) no es más fuerte que \(p\).
\vspace{-0.3cm}
    \item \(p\) es más fuerte que \(p \rightarrow q\).
    \vspace{-0.3cm}
    \item \(p\) no es más fuerte que \(q\).
    \vspace{-0.3cm}
    \item \textbf{False} es la fórmula más fuerte de todas.
    \vspace{-0.3cm}
    \item \textbf{True} es la fórmula más débil de todas.
    \vspace{-0.3cm}
\end{itemize}
\subsection*{\tiny {Truco algebraico para sacar términos del último rango}}
    \vspace{-0.2cm}
(\forall k:\ent) \, (i \leq k < |s| \implicaLuego p(k)) \equiv \land p(i) (\forall k:\ent ) (i < k < |s|) \implicaLuego (p(k))

Por otro lado, para sumatorias, tenemos que:
\sum_{k=i}^{|s|-1} p(k) = p(i) + \sum_{k=i+1}^{|s|-1} p(k)
\\ ejemplo:
\[ \sum_{j=0}^{5} j = \sum_{j=0}^{4} j + 5 \] \\
Esto nos permite sacar el término correspondiente a \( i \) fuera de la sumatoria.




\end{document}




% Reducir tamaño de fuente para tablas
\newcommand{\smalltable}{\fontsize{8pt}{10pt}\selectfont}
\renewcommand{\arraystretch}{0.8}  % Reduce el espacio vertical en las tablas.


\begin{document}

\section*{\tiny{Tipos complejos}}
    \vspace{-0.2cm}
\subsection*{\tiny{seq$<T>$: secuencia de tipo T}}
\vspace{-0.2cm}
\noindent\smalltable
\begin{tabularx}{\linewidth}{|X|X|}
\hline
crear & \( \langle \rangle \), \( \langle x, y, z \rangle \) \\
\hline
tamaño & \( |s| \), \text{length}(s) \\
\hline
pertenece & \( i \in s \) \\
\hline
ver posición & \( s[i] \) \\
\hline
cabeza & \text{head}(s) \\
\hline
cola & \text{tail}(s) \\
\hline
concatenar & \text{concat}(s1, s2), \( s1 + s2 \) \\
\hline
subsecuencia & \text{subseq}(s, i, j), \( s[i..j] \) \\
\hline
setear posición & \text{setAt}(s, i, \text{val}) \\
\hline
\end{tabularx}
\vspace{-0.5cm}
\subsection*{\tiny{conj$<T>$: conjunto de tipo T}}
\vspace{-0.2cm}
\noindent\smalltable
\vspace{-0.4cm}
\begin{tabularx}{\linewidth}{|X|X|}
\hline
crear & \( \{\} \), \( \{x, y, z\} \) \\
\hline
tamaño & \( |c| \), \text{length}(c) \\
\hline
pertenece & \( i \in c \) \\
\hline
unión & \( c1 \cup c2 \) \\
\hline
intersección & \( c1 \cap c2 \) \\
\hline
diferencia & \( c1 - c2 \) \\
\hline
\end{tabularx}
\vspace{-0.1cm}
\subsection*{\tiny{dict$<$K, V$>$: diccionario que asocia claves de tipo K con valores de tipo V}}
\vspace{-0.2cm}
\noindent\smalltable
\vspace{-0.4cm}
\begin{tabularx}{\linewidth}{|X|X|}
\hline
crear & \( \{\} \), \( \{\text{``juan''} : 20, \text{``diego''} : 10\} \) \\
\hline
tamaño & \( |d| \), \text{length}(d) \\
\hline
pertenece (hay clave) & \( k \in d \) \\
\hline
valor & d[k] \\
\hline
setear valor & \text{setKey}(d, k, v) \\
\hline
eliminar valor & \text{delKey}(d, k) \\
\hline
\end{tabularx}
\vspace{-0.1cm}
\subsection*{\tiny{tupla$<$T1, ..., Tn$>$: tupla de tipos T1, ..., Tn}}
\vspace{-0.2cm}
\noindent\smalltable
\begin{tabularx}{\linewidth}{|X|X|}
\hline
crear & \( \langle x, y, z \rangle \) \\
\hline
campo & \( s[i] \) \\
\hline
\end{tabularx}
\vspace{-0.5cm}
\subsection*{\tiny{struct$<$campo1: T1, ..., campon: Tn$>$: tupla con nombres para los campos.}}
\vspace{-0.2cm}
\noindent\smalltable
\begin{tabularx}{\linewidth}{|X|X|}
\hline
crear & \( \langle x : 20, y : 10 \rangle \) \\
\hline
campo & \( s_x, s_y \) \\
\hline
\end{tabularx}
\vspace{-0.2cm}
\subsection*{\tiny{Precondición más débil (wp)}}
\vspace{-0.3cm}
\noindent\smalltable
\vspace{-0.2cm}
La \textit{precondición más débil} es la condición mínima necesaria antes de ejecutar una sentencia para garantizar que una postcondición dada se cumpla.
\subsection*{\tiny{Lógica de Hoare}}
\vspace{-0.3cm}
\noindent\smalltable
\vspace{-0.1cm}
Una forma de razonar sobre la corrección de programas es a través de las triples de Hoare \(\{P\} C \{Q\}\). El objetivo es obtener una fórmula lógica \( \alpha \) tal que \( \alpha \) es verdadera si y solo si \(\{P\} C \{Q\}\) es verdadero.
\vspace{-0.5cm}
\paragraph*{Predicado def(E):}
Definición. Dada una expresión \( E \), llamamos \( \text{def}(E) \) a las condiciones necesarias para que \( E \) esté definida.
\vspace{-0.45cm}
\paragraph*{Predicado \( Q_{E}^x \):}
Definición. Dado un predicado \( Q \), el predicado \( Q_{E}^x \) se obtiene reemplazando...
\vspace{-0.4cm}
\paragraph*{Axiomas WP}
\begin{itemize}\vspace{-0.35cm}
    \item \textbf{Axioma 1.} \( \text{wp}(x := E, Q) \equiv \text{def}(E) \land Q_{E}^x \)\vspace{-0.35cm}
    \item \textbf{Axioma 2.} \( \text{wp}(\text{skip}, Q) \equiv Q \).
    \vspace{-0.33cm}
    \item \textbf{Axioma 3.} \( \text{wp}(S1; S2, Q) \equiv \text{wp}(S1,\text{wp}(S2, Q)) \).\vspace{-0.33cm}
    \item \textbf{Axioma 4.} Si \(S = \text{if } B \text{ then } S1 \text{ else } S2 \text{ endif}\), entonces
    \vspace{-0.31cm}
    \[
    \vspace{-0.37cm}
    \text{wp}(S, Q) \equiv \text{def}(B) \yLuego \left( (B \land \text{wp}(S1, Q)) \vee (\neg B \land \text{wp}(S2, Q)) \right)
    \]
     \item \textbf{Axioma 5.} \( \text{wp}(\text{while } B \text{ do } S \text{ endwhile}, Q) \equiv (\exists_{i \geq 0})({ H_i}(Q)) \)  no podemos usar mecanicamente el Axioma 5
para demostrar la corrección de un ciclo con una cantidad no
acotada a priori de iteraciones
\vspace{-0.34cm}
\end{itemize}
\subsection*{\tiny{Asignación a elementos de una secuencia}}
\noindent\smalltable
\vspace{-0.3cm}
Aplicando el Axioma 1, obtenemos:
\[ \text{wp}(b[i] := E, Q) \equiv \text{wp}(b := \text{setAt}(b, i, E), Q) \]
\[ \equiv (\text{def}(b) \land \text{def}(i)) \land 0 \leq i < |b| \land \text{def}(E) \land Q_{\text{setAt}(b,i,E)}^b \]

Dado que \( 0 \leq i, j < |b| \), se sabe que:
\[ \text{setAt}(b, i, E)[j] = 
\begin{cases} 
\vspace{-0.2cm}
E & \text{si } i = j \\
b[j] & \text{si } i \neq j 
\end{cases} 
\]
\textbf{Ejemplo.}
\[ \text{wp}(b[i] := 5, b[j] = 2) \]
\[ \equiv \text{setAt}(b, i, 5)[j] = 2 \]
\[ \equiv (i \neq j \land \text{setAt}(b, i, 5)[j] = 2) \lor (i = j \land \text{setAt}(b, i, 5)[j] = 2) \]
\[ \equiv (i \neq j \land b[j] = 2) \lor (i = j \land \text{setAt}(b, i, 5)[i] = 2) \]
\[ \equiv (i \neq j \land b[j] = 2) \lor (i = j \land 5 = 2) \]
\[ \equiv i \neq j \land b[j] = 2 \]

\vspace{-0.4cm}
\vspace{-0.5cm}
\paragraph*{Propiedades:}
\begin{itemize}
\vspace{-0.32cm}
    \item Monotonía: Si \( Q \) implica \( R \), entonces \( \text{wp}(S, Q) \) implica \( \text{wp}(S, R) \).
    \vspace{-0.32cm}
    \item Distributividad: \( \text{wp}(S, Q \land R) \) equivale a \( \text{wp}(S, Q) \land \text{wp}(S, R) \).
    \vspace{-0.32cm}
    \item Excluded Miracle: \( \text{wp}(S, \text{false}) \) equivale a \( \text{false} \).
\end{itemize}
\vspace{-0.67cm}
\subsection*{\tiny{Invariante de un ciclo}}
\noindent\smalltable
\vspace{-0.2cm}
\textbf{Definición.} Un predicado \( I \) es un invariante de un ciclo si:
\begin{enumerate}
    \item \( I \) vale antes de comenzar el ciclo, y
    \vspace{-0.32cm}
    \item Si vale \( I \land B \) al comenzar una iteración arbitraria, entonces sigue valiendo \( I \) al finalizar la ejecución del cuerpo del ciclo.
\end{enumerate}
\vspace{-0.22cm}
Un invariante describe un estado que se satisface cada vez que comienza la ejecución del cuerpo de un ciclo y también se cumple cuando la ejecución del cuerpo del ciclo concluye.
\vspace{-0.2cm}
\textbf{Teorema del Invariante.} Si existe un predicado \( I \) tal que:
\begin{enumerate}
    \item \( P_C \Rightarrow I \),
    \item $ \{ I \wedge B \} S \{ I \} \iff (I \wedge B) \implies wp(S, I) $
    \item \(I \land \neg B \Rightarrow Q_C\),
\end{enumerate}
entonces \( \{P_C\} \text{ while } B \text{ do } S \text{ endwhile } \{Q_C\} \) es válida.
\subsection*{\tiny{Función variante}}
\noindent\smalltable
La función variante representa una cantidad que se va
reduciendo
\begin{itemize}
    \item {Si existe una función variante \(f_v\) tal que:
\begin{enumerate}\item $ \{ I \wedge B \wedge f_v = v_0 \} S \{ f_v < v_0 \} \iff (I \wedge B \wedge f_v = v_0) \implies wp(S, f_v < v_0) $

    \item Si \(I \land f_v \leq 0\) implica \(\neg B\)
\end{enumerate}}
\end{itemize}
\subsection*{\tiny{Correctitud de un Programa Completo}}
\noindent\smalltable
Para demostrar la correctitud de un programa completo utilizando la lógica de Hoare, seguimos estos pasos:

\begin{enumerate}
    \item \textbf{Código antes del ciclo:} Debemos demostrar que las precondiciones implican la precondición más débil del código previo al ciclo, es decir,
    \[ Pre \Rightarrow \text{wp}(Código\_previo, P_c) \]
    
    \item \textbf{Correctitud del ciclo:} Utilizamos el teorema del invariante, ya que no podemos calcular la precondición más débil del ciclo en general. El teorema del invariante se expresa de la siguiente manera:
    
    \begin{enumerate}
        \item \[ P_c \Rightarrow I \]
        \item \[ (I \land B) \Rightarrow \text{wp}(Ciclo, I) \]
        \item \[ (I \land \neg B) \Rightarrow Q_c \]
    \end{enumerate}

    \item \textbf{Código posterior al ciclo:} Comprobamos que las postcondiciones del ciclo implican la precondición más débil del código posterior al ciclo, es decir,
    \[ Q_c \Rightarrow \text{wp}(Código\_posterior, Post) \]

    \item \textbf{Conclusión:} Si probamos estas tres cosas, podemos concluir, por corolario de monotonía, que el programa completo es correcto con respecto a la especificación:
    \[ Pre \Rightarrow \text{wp}(Programa\_completo, Post) \]
\end{enumerate}
\vspace{-0.6cm}

\subsection*{\tiny{Especificación y Relaciones de Fuerza}}

\noindent\textbf{Especificación:} \\
La especificación define qué es lo que debe hacer un algoritmo, en términos de relación entre sus entradas y sus salidas, sin determinar cómo lo hace. Las relaciones de fuerza entre especificaciones son importantes para entender cuán restrictiva o general es una especificación con respecto a otra.

\noindent\textbf{Subespecificación:} \\
Implica otorgar una precondición más restrictiva o una postcondición más débil que lo deducido del enunciado del problema. Una precondición más restrictiva excluye casos de entrada posibles. Una postcondición más débil permite soluciones no deseadas.

\noindent\textbf{Sobreespecificación:} \\
Consiste en proporcionar una postcondición más restrictiva o una precondición más débil que lo necesario. Una precondición más débil obliga al algoritmo a considerar casos innecesarios. Una postcondición más restrictiva limita las posibles soluciones.

\noindent\textbf{Relaciones de Fuerza:} \\
Decimos que \(A\) es más fuerte que \(B\) cuando \(A \rightarrow B\) es una tautología. También podemos afirmar que \(A\) fuerza a \(B\) o que \(B\) es más débil que \(A\).

\noindent\textbf{Ejemplos:}
\begin{itemize}
\vspace{-0.3cm}
    \item \(p \land q\) es más fuerte que \(p\).
    \vspace{-0.3cm}
    \item \(p \lor q\) no es más fuerte que \(p\).
\vspace{-0.3cm}
    \item \(p\) es más fuerte que \(p \rightarrow q\).
    \vspace{-0.3cm}
    \item \(p\) no es más fuerte que \(q\).
    \vspace{-0.3cm}
    \item \textbf{False} es la fórmula más fuerte de todas.
    \vspace{-0.3cm}
    \item \textbf{True} es la fórmula más débil de todas.
    \vspace{-0.3cm}
\end{itemize}
\subsection*{\tiny {Truco algebraico para sacar términos del último rango}}
    \vspace{-0.2cm}
(\forall k:\ent) \, (i \leq k < |s| \implicaLuego p(k)) \equiv p(i) \wedge (\forall k:\ent ) (i < k < |s|) \implicaLuego (p(k))

Por otro lado, para sumatorias, tenemos que:
\sum_{k=i}^{|s|-1} p(k) = p(i) + \sum_{k=i+1}^{|s|-1} p(k)
\\ ejemplo:
\[ \sum_{j=0}^{5} j = \sum_{j=0}^{4} j + 5 \] \\
Esto nos permite sacar el término correspondiente a \( i \) fuera de la sumatoria.




\end{document}


